% Options for packages loaded elsewhere
\PassOptionsToPackage{unicode}{hyperref}
\PassOptionsToPackage{hyphens}{url}
\documentclass[
]{article}
\usepackage{xcolor}
\usepackage[margin=1in]{geometry}
\usepackage{amsmath,amssymb}
\setcounter{secnumdepth}{-\maxdimen} % remove section numbering
\usepackage{iftex}
\ifPDFTeX
  \usepackage[T1]{fontenc}
  \usepackage[utf8]{inputenc}
  \usepackage{textcomp} % provide euro and other symbols
\else % if luatex or xetex
  \usepackage{unicode-math} % this also loads fontspec
  \defaultfontfeatures{Scale=MatchLowercase}
  \defaultfontfeatures[\rmfamily]{Ligatures=TeX,Scale=1}
\fi
\usepackage{lmodern}
\ifPDFTeX\else
  % xetex/luatex font selection
\fi
% Use upquote if available, for straight quotes in verbatim environments
\IfFileExists{upquote.sty}{\usepackage{upquote}}{}
\IfFileExists{microtype.sty}{% use microtype if available
  \usepackage[]{microtype}
  \UseMicrotypeSet[protrusion]{basicmath} % disable protrusion for tt fonts
}{}
\makeatletter
\@ifundefined{KOMAClassName}{% if non-KOMA class
  \IfFileExists{parskip.sty}{%
    \usepackage{parskip}
  }{% else
    \setlength{\parindent}{0pt}
    \setlength{\parskip}{6pt plus 2pt minus 1pt}}
}{% if KOMA class
  \KOMAoptions{parskip=half}}
\makeatother
\usepackage{color}
\usepackage{fancyvrb}
\newcommand{\VerbBar}{|}
\newcommand{\VERB}{\Verb[commandchars=\\\{\}]}
\DefineVerbatimEnvironment{Highlighting}{Verbatim}{commandchars=\\\{\}}
% Add ',fontsize=\small' for more characters per line
\usepackage{framed}
\definecolor{shadecolor}{RGB}{248,248,248}
\newenvironment{Shaded}{\begin{snugshade}}{\end{snugshade}}
\newcommand{\AlertTok}[1]{\textcolor[rgb]{0.94,0.16,0.16}{#1}}
\newcommand{\AnnotationTok}[1]{\textcolor[rgb]{0.56,0.35,0.01}{\textbf{\textit{#1}}}}
\newcommand{\AttributeTok}[1]{\textcolor[rgb]{0.13,0.29,0.53}{#1}}
\newcommand{\BaseNTok}[1]{\textcolor[rgb]{0.00,0.00,0.81}{#1}}
\newcommand{\BuiltInTok}[1]{#1}
\newcommand{\CharTok}[1]{\textcolor[rgb]{0.31,0.60,0.02}{#1}}
\newcommand{\CommentTok}[1]{\textcolor[rgb]{0.56,0.35,0.01}{\textit{#1}}}
\newcommand{\CommentVarTok}[1]{\textcolor[rgb]{0.56,0.35,0.01}{\textbf{\textit{#1}}}}
\newcommand{\ConstantTok}[1]{\textcolor[rgb]{0.56,0.35,0.01}{#1}}
\newcommand{\ControlFlowTok}[1]{\textcolor[rgb]{0.13,0.29,0.53}{\textbf{#1}}}
\newcommand{\DataTypeTok}[1]{\textcolor[rgb]{0.13,0.29,0.53}{#1}}
\newcommand{\DecValTok}[1]{\textcolor[rgb]{0.00,0.00,0.81}{#1}}
\newcommand{\DocumentationTok}[1]{\textcolor[rgb]{0.56,0.35,0.01}{\textbf{\textit{#1}}}}
\newcommand{\ErrorTok}[1]{\textcolor[rgb]{0.64,0.00,0.00}{\textbf{#1}}}
\newcommand{\ExtensionTok}[1]{#1}
\newcommand{\FloatTok}[1]{\textcolor[rgb]{0.00,0.00,0.81}{#1}}
\newcommand{\FunctionTok}[1]{\textcolor[rgb]{0.13,0.29,0.53}{\textbf{#1}}}
\newcommand{\ImportTok}[1]{#1}
\newcommand{\InformationTok}[1]{\textcolor[rgb]{0.56,0.35,0.01}{\textbf{\textit{#1}}}}
\newcommand{\KeywordTok}[1]{\textcolor[rgb]{0.13,0.29,0.53}{\textbf{#1}}}
\newcommand{\NormalTok}[1]{#1}
\newcommand{\OperatorTok}[1]{\textcolor[rgb]{0.81,0.36,0.00}{\textbf{#1}}}
\newcommand{\OtherTok}[1]{\textcolor[rgb]{0.56,0.35,0.01}{#1}}
\newcommand{\PreprocessorTok}[1]{\textcolor[rgb]{0.56,0.35,0.01}{\textit{#1}}}
\newcommand{\RegionMarkerTok}[1]{#1}
\newcommand{\SpecialCharTok}[1]{\textcolor[rgb]{0.81,0.36,0.00}{\textbf{#1}}}
\newcommand{\SpecialStringTok}[1]{\textcolor[rgb]{0.31,0.60,0.02}{#1}}
\newcommand{\StringTok}[1]{\textcolor[rgb]{0.31,0.60,0.02}{#1}}
\newcommand{\VariableTok}[1]{\textcolor[rgb]{0.00,0.00,0.00}{#1}}
\newcommand{\VerbatimStringTok}[1]{\textcolor[rgb]{0.31,0.60,0.02}{#1}}
\newcommand{\WarningTok}[1]{\textcolor[rgb]{0.56,0.35,0.01}{\textbf{\textit{#1}}}}
\usepackage{longtable,booktabs,array}
\usepackage{calc} % for calculating minipage widths
% Correct order of tables after \paragraph or \subparagraph
\usepackage{etoolbox}
\makeatletter
\patchcmd\longtable{\par}{\if@noskipsec\mbox{}\fi\par}{}{}
\makeatother
% Allow footnotes in longtable head/foot
\IfFileExists{footnotehyper.sty}{\usepackage{footnotehyper}}{\usepackage{footnote}}
\makesavenoteenv{longtable}
\usepackage{graphicx}
\makeatletter
\newsavebox\pandoc@box
\newcommand*\pandocbounded[1]{% scales image to fit in text height/width
  \sbox\pandoc@box{#1}%
  \Gscale@div\@tempa{\textheight}{\dimexpr\ht\pandoc@box+\dp\pandoc@box\relax}%
  \Gscale@div\@tempb{\linewidth}{\wd\pandoc@box}%
  \ifdim\@tempb\p@<\@tempa\p@\let\@tempa\@tempb\fi% select the smaller of both
  \ifdim\@tempa\p@<\p@\scalebox{\@tempa}{\usebox\pandoc@box}%
  \else\usebox{\pandoc@box}%
  \fi%
}
% Set default figure placement to htbp
\def\fps@figure{htbp}
\makeatother
\setlength{\emergencystretch}{3em} % prevent overfull lines
\providecommand{\tightlist}{%
  \setlength{\itemsep}{0pt}\setlength{\parskip}{0pt}}
\usepackage{bookmark}
\IfFileExists{xurl.sty}{\usepackage{xurl}}{} % add URL line breaks if available
\urlstyle{same}
\hypersetup{
  pdftitle={Raster Polygonizer Workflow},
  hidelinks,
  pdfcreator={LaTeX via pandoc}}

\title{Raster Polygonizer Workflow}
\author{}
\date{\vspace{-2.5em}}

\begin{document}
\maketitle

\section{From LiDAR Scan to Building
Footprint}\label{from-lidar-scan-to-building-footprint}

\emph{A complete workflow for extracting, filtering, and characterizing
rooftop polygons from aerial elevation data}

\begin{center}\rule{0.5\linewidth}{0.5pt}\end{center}

\subsection{Why This Package?}\label{why-this-package}

Heavy snow accumulation on rooftops is a serious safety concern in cold
climates --- when loads grow too large, roofs can fail catastrophically.
Understanding and predicting this risk requires detailed information
about building geometry across entire cities, but traditional manual
surveys are slow, expensive, and limited in scale.

Airborne LiDAR (Light Detection and Ranging) offers a powerful
alternative. By bouncing laser pulses off surfaces from an aircraft and
recording the return time, LiDAR produces dense, precise
three-dimensional maps of the built environment. Comparing scans taken
before and after a snowfall event makes it possible to measure rooftop
snow accumulation automatically --- at city scale, for thousands of
buildings at once.

\texttt{rasterpolygonizer} packages the full processing chain needed to
go from a raw LiDAR-derived elevation raster to a clean, attributed set
of building footprints, ready for snow load analysis, engineering
assessment, or any other rooftop-scale application.

\begin{center}\rule{0.5\linewidth}{0.5pt}\end{center}

\subsection{Workflow Overview}\label{workflow-overview}

The six functions in this package form a linear pipeline. Each step
produces output consumed by the next:

\begin{longtable}[]{@{}
  >{\centering\arraybackslash}p{(\linewidth - 4\tabcolsep) * \real{0.2400}}
  >{\raggedright\arraybackslash}p{(\linewidth - 4\tabcolsep) * \real{0.4000}}
  >{\raggedright\arraybackslash}p{(\linewidth - 4\tabcolsep) * \real{0.3600}}@{}}
\toprule\noalign{}
\begin{minipage}[b]{\linewidth}\centering
Step
\end{minipage} & \begin{minipage}[b]{\linewidth}\raggedright
Function
\end{minipage} & \begin{minipage}[b]{\linewidth}\raggedright
Purpose
\end{minipage} \\
\midrule\noalign{}
\endhead
\bottomrule\noalign{}
\endlastfoot
1 & \texttt{fill\_ground\_raster()} & Fill voids using clustering-based
ground estimation \\
2 & \texttt{extract\_building\_edges\_to\_polygons()} & Detect edges via
Sobel gradients and close them morphologically \\
3 & \texttt{clean\_building\_polygons()} & Polygonize enclosed
interiors, then shrink, simplify, and filter by area \\
4 & \texttt{filter\_by\_ground\_truth()} & Retain candidates with
sufficient containment within reference footprints \\
5 & \texttt{remove\_invalid\_polys()} & Remove ground-contaminated and
nested polygons \\
6a & \texttt{estimate\_building\_height()} & Estimate height relative to
surrounding ground elevation \\
6b & \texttt{roof\_slope\_RANSAC()} & Fit robust planes and derive slope
and aspect per roof \\
\end{longtable}

Steps 6a and 6b are independent characterization steps --- run either or
both depending on your analysis goals.

\begin{center}\rule{0.5\linewidth}{0.5pt}\end{center}

\subsection{Load Example Data}\label{load-example-data}

\begin{Shaded}
\begin{Highlighting}[]
\FunctionTok{library}\NormalTok{(ggplot2)}

\NormalTok{load\_extdata }\OtherTok{\textless{}{-}} \ControlFlowTok{function}\NormalTok{(filename, }\AttributeTok{pkg =} \StringTok{"rasterpolygonizer"}\NormalTok{) \{}
\NormalTok{  path }\OtherTok{\textless{}{-}} \FunctionTok{system.file}\NormalTok{(}\StringTok{"extdata"}\NormalTok{, filename, }\AttributeTok{package =}\NormalTok{ pkg)}
  \ControlFlowTok{if}\NormalTok{ (path }\SpecialCharTok{==} \StringTok{""}\NormalTok{) }\FunctionTok{stop}\NormalTok{(}\FunctionTok{sprintf}\NormalTok{(}\StringTok{"File \textquotesingle{}\%s\textquotesingle{} not found in inst/extdata"}\NormalTok{, filename))}
\NormalTok{  path}
\NormalTok{\}}

\NormalTok{r }\OtherTok{\textless{}{-}}\NormalTok{ terra}\SpecialCharTok{::}\FunctionTok{rast}\NormalTok{(}\FunctionTok{load\_extdata}\NormalTok{(}\StringTok{"sample\_raster.tif"}\NormalTok{))}
\NormalTok{b }\OtherTok{\textless{}{-}}\NormalTok{ sf}\SpecialCharTok{::}\FunctionTok{st\_read}\NormalTok{(}\FunctionTok{load\_extdata}\NormalTok{(}\StringTok{"sample\_buildings.gpkg"}\NormalTok{), }\AttributeTok{quiet =} \ConstantTok{TRUE}\NormalTok{)}
\end{Highlighting}
\end{Shaded}

The sample dataset covers a small area of Fairbanks, Alaska. \texttt{r}
is a LiDAR-derived digital surface model (DSM) --- a raster where each
cell records the elevation of the highest surface at that location,
whether bare earth, a rooftop, or a tree. \texttt{b} contains digitized
building footprints provided by the Fairbanks North Star Borough.

\begin{center}\rule{0.5\linewidth}{0.5pt}\end{center}

\#We define a shared ggplot theme using a warm earth-tone document
background, and a blue-to-gold raster fill palette. Blue and gold sit
across from each other on the color wheel --- the cool-to-warm contrast
gives elevation data strong visual pop against the neutral document
background while remaining readable in print.

\subsection{Step 1 --- Fill the Ground
Raster}\label{step-1-fill-the-ground-raster}

\textbf{\texttt{fill\_ground\_raster(raster,\ w\ =\ 15)}}

LiDAR point clouds frequently contain data voids where the sensor
received no usable return, typically due to occlusion between the
collection device and the point of measurement. Left unfilled, these
holes disrupt the gradient computation in Step 2.

\texttt{fill\_ground\_raster()} addresses this with a focal neighborhood
approach designed specifically for elevation data. For each NA cell, it
examines all valid neighbors within a \texttt{w\ ×\ w} window and uses
k-means clustering (up to 3 clusters) to separate them by elevation. It
then returns the median of the \emph{lowest} cluster --- the best
available estimate of ground level at that location. This means the fill
preferentially recovers ground-level values rather than being pulled
upward by nearby rooftops or vegetation, which is exactly what
downstream processing needs.

\textbf{\texttt{w}} (default \texttt{15}): The focal window size in
pixels. Larger windows can fill bigger voids but may pull values from
more distant and less representative neighbors. Typically this window
size will work well unless the buildings of interest are very large.

\begin{Shaded}
\begin{Highlighting}[]
\NormalTok{r\_df\_raw }\OtherTok{\textless{}{-}} \FunctionTok{as.data.frame}\NormalTok{(r, }\AttributeTok{xy =} \ConstantTok{TRUE}\NormalTok{, }\AttributeTok{na.rm =} \ConstantTok{TRUE}\NormalTok{)}
\FunctionTok{names}\NormalTok{(r\_df\_raw)[}\DecValTok{3}\NormalTok{] }\OtherTok{\textless{}{-}} \StringTok{"value"}

\FunctionTok{ggplot}\NormalTok{(r\_df\_raw) }\SpecialCharTok{+}
  \FunctionTok{geom\_raster}\NormalTok{(}\FunctionTok{aes}\NormalTok{(}\AttributeTok{x =}\NormalTok{ x, }\AttributeTok{y =}\NormalTok{ y, }\AttributeTok{fill =}\NormalTok{ value)) }\SpecialCharTok{+}
  \FunctionTok{coord\_equal}\NormalTok{() }\SpecialCharTok{+}
  \FunctionTok{scale\_fill\_rp}\NormalTok{(}\AttributeTok{name =} \StringTok{"Elev. (m)"}\NormalTok{) }\SpecialCharTok{+}
  \FunctionTok{theme\_rp}\NormalTok{() }\SpecialCharTok{+}
  \FunctionTok{labs}\NormalTok{(}
    \AttributeTok{title    =} \StringTok{"Raw Input Raster"}\NormalTok{,}
    \AttributeTok{subtitle =} \StringTok{"LiDAR digital surface model — voids appear where returns were absent or filtered"}
\NormalTok{  )}
\end{Highlighting}
\end{Shaded}

\begin{center}\includegraphics{raster-polygonizer-workflow_files/figure-latex/fill-1} \end{center}

\begin{Shaded}
\begin{Highlighting}[]
\NormalTok{r\_filled }\OtherTok{\textless{}{-}} \FunctionTok{fill\_ground\_raster}\NormalTok{(r)}

\NormalTok{r\_df\_filled }\OtherTok{\textless{}{-}} \FunctionTok{as.data.frame}\NormalTok{(r\_filled, }\AttributeTok{xy =} \ConstantTok{TRUE}\NormalTok{, }\AttributeTok{na.rm =} \ConstantTok{TRUE}\NormalTok{)}
\FunctionTok{names}\NormalTok{(r\_df\_filled)[}\DecValTok{3}\NormalTok{] }\OtherTok{\textless{}{-}} \StringTok{"value"}

\FunctionTok{ggplot}\NormalTok{(r\_df\_filled) }\SpecialCharTok{+}
  \FunctionTok{geom\_raster}\NormalTok{(}\FunctionTok{aes}\NormalTok{(}\AttributeTok{x =}\NormalTok{ x, }\AttributeTok{y =}\NormalTok{ y, }\AttributeTok{fill =}\NormalTok{ value)) }\SpecialCharTok{+}
  \FunctionTok{coord\_equal}\NormalTok{() }\SpecialCharTok{+}
  \FunctionTok{scale\_fill\_rp}\NormalTok{(}\AttributeTok{name =} \StringTok{"Elev. (m)"}\NormalTok{) }\SpecialCharTok{+}
  \FunctionTok{theme\_rp}\NormalTok{() }\SpecialCharTok{+}
  \FunctionTok{labs}\NormalTok{(}
    \AttributeTok{title    =} \StringTok{"Filled Ground Raster"}\NormalTok{,}
    \AttributeTok{subtitle =} \StringTok{"Voids filled using clustering{-}based ground estimation"}
\NormalTok{  )}
\end{Highlighting}
\end{Shaded}

\begin{center}\includegraphics{raster-polygonizer-workflow_files/figure-latex/fill-2} \end{center}

\begin{center}\rule{0.5\linewidth}{0.5pt}\end{center}

\subsection{Step 2 --- Extract Building Edge
Polygons}\label{step-2-extract-building-edge-polygons}

\textbf{\texttt{extract\_building\_edges\_to\_polygons(raster,\ smooth\_w,\ thr\_prob,\ morph\_w)}}

Buildings appear in a DSM as abrupt vertical discontinuities --- sharp
jumps in elevation from ground to rooftop. This function detects those
discontinuities through a four-stage image processing pipeline:

\begin{enumerate}
\def\labelenumi{\arabic{enumi}.}
\tightlist
\item
  \textbf{Smoothing} --- a mean focal filter reduces sensor noise before
  gradient computation
\item
  \textbf{Sobel gradient} --- horizontal (Gx) and vertical (Gy) Sobel
  kernels are applied to the smoothed raster, and their combined
  magnitude (\texttt{sqrt(Gx²\ +\ Gy²)}) captures edge strength in all
  directions
\item
  \textbf{Thresholding} --- pixels whose gradient magnitude exceeds the
  \texttt{thr\_prob} quantile of all gradient values are classified as
  edges
\item
  \textbf{Morphological closing} --- a expand-then-erode operation
  bridges small gaps in the detected edge lines, producing closed loops
  that surround building interiors
\end{enumerate}

\textbf{\texttt{thr\_prob}} (default \texttt{0.70}): The quantile
threshold applied to gradient magnitude. At \texttt{0.70}, the top 30\%
of gradient values are classified as edges. \emph{Raising this value
makes the detector more selective} --- only the strongest gradients
(sharpest elevation transitions) pass through, which helps in dense
urban scenes with many small buildings where vegetation and clutter
produce weaker false edges. Lowering it recovers subtler edges but
increases noise.

\begin{Shaded}
\begin{Highlighting}[]
\NormalTok{edges }\OtherTok{\textless{}{-}} \FunctionTok{extract\_building\_edges\_to\_polygons}\NormalTok{(r\_filled, }\AttributeTok{thr\_prob =} \FloatTok{0.8}\NormalTok{)}
\end{Highlighting}
\end{Shaded}

\begin{center}\rule{0.5\linewidth}{0.5pt}\end{center}

\subsection{Step 3 --- Clean Building
Polygons}\label{step-3-clean-building-polygons}

\textbf{\texttt{clean\_building\_polygons(closed\_edges,\ shrink\_dist,\ simplify\_tol,\ min\_area,\ max\_area)}}

The closed edge loops from Step 2 define the \emph{perimeters} of
buildings, but what we actually need are filled polygon footprints. The
key insight in this function is that the edge loops enclose interior
regions --- so the function first \emph{inverts} the edge raster and
polygonizes the enclosed FALSE (interior) regions. Those interiors are
the building footprints.

After polygonization, three cleaning operations are applied in sequence:

\textbf{\texttt{shrink\_dist}} (default \texttt{-0.5}, map units): A
negative buffer shrinks each polygon inward. This is necessary because
the edge band detected in Step 2 sits slightly outside the true rooftop
boundary --- the shrink compensates for that outward bias. Use a more
negative value for coarser-resolution rasters where the edge band is
wider.

\textbf{\texttt{simplify\_tol}} (default \texttt{0.5}, map units):
Douglas-Peucker simplification removes redundant vertices and smooths
jagged outlines without distorting the overall footprint shape. Larger
values produce cleaner but blockier polygons.

\textbf{\texttt{min\_area} / \texttt{max\_area}}: After exploding all
multi-part geometries into individual polygons, those outside this size
range are dropped. \texttt{min\_area} removes noise patches and the
small gaps between edge pixels; \texttt{max\_area} removes the large
background region (the inverted raster always produces one polygon
covering everything \emph{outside} the buildings, which is discarded
here).

The function also computes and attaches \texttt{area},
\texttt{perimeter}, \texttt{perim\_area\_ratio}, and \texttt{vertices}
columns to the output for downstream use.

\begin{Shaded}
\begin{Highlighting}[]
\NormalTok{buildings\_sf }\OtherTok{\textless{}{-}} \FunctionTok{clean\_building\_polygons}\NormalTok{(}
  \AttributeTok{closed\_edges =}\NormalTok{ edges}\SpecialCharTok{$}\NormalTok{closed\_edges,}
  \AttributeTok{shrink\_dist  =} \SpecialCharTok{{-}}\FloatTok{0.5}\NormalTok{,}
  \AttributeTok{simplify\_tol =} \FloatTok{2.5}\NormalTok{,}
  \AttributeTok{min\_area     =} \FloatTok{19.99}\NormalTok{,}
  \AttributeTok{max\_area     =} \DecValTok{3000}
\NormalTok{)}

\NormalTok{r\_df }\OtherTok{\textless{}{-}} \FunctionTok{as.data.frame}\NormalTok{(r\_filled, }\AttributeTok{xy =} \ConstantTok{TRUE}\NormalTok{, }\AttributeTok{na.rm =} \ConstantTok{TRUE}\NormalTok{)}
\FunctionTok{names}\NormalTok{(r\_df)[}\DecValTok{3}\NormalTok{] }\OtherTok{\textless{}{-}} \StringTok{"value"}

\FunctionTok{ggplot}\NormalTok{() }\SpecialCharTok{+}
  \FunctionTok{geom\_raster}\NormalTok{(}\AttributeTok{data =}\NormalTok{ r\_df, }\FunctionTok{aes}\NormalTok{(}\AttributeTok{x =}\NormalTok{ x, }\AttributeTok{y =}\NormalTok{ y, }\AttributeTok{fill =}\NormalTok{ value)) }\SpecialCharTok{+}
  \FunctionTok{geom\_sf}\NormalTok{(}\AttributeTok{data =}\NormalTok{ buildings\_sf, }\AttributeTok{fill =} \ConstantTok{NA}\NormalTok{, }\AttributeTok{color =} \StringTok{"\#c47c0f"}\NormalTok{, }\AttributeTok{linewidth =} \FloatTok{0.6}\NormalTok{) }\SpecialCharTok{+}
  \FunctionTok{coord\_sf}\NormalTok{() }\SpecialCharTok{+}
  \FunctionTok{scale\_fill\_rp}\NormalTok{(}\AttributeTok{name =} \StringTok{"Elev. (m)"}\NormalTok{) }\SpecialCharTok{+}
  \FunctionTok{theme\_rp}\NormalTok{() }\SpecialCharTok{+}
  \FunctionTok{labs}\NormalTok{(}
    \AttributeTok{title    =} \StringTok{"Candidate Building Polygons"}\NormalTok{,}
    \AttributeTok{subtitle =} \StringTok{"Interior regions enclosed by edge loops, shrunk and simplified"}
\NormalTok{  )}
\end{Highlighting}
\end{Shaded}

\begin{center}\includegraphics{raster-polygonizer-workflow_files/figure-latex/clean-1} \end{center}

\begin{center}\rule{0.5\linewidth}{0.5pt}\end{center}

\subsection{Step 4 --- Filter by Ground
Truth}\label{step-4-filter-by-ground-truth}

\textbf{\texttt{filter\_by\_ground\_truth(candidates,\ truth,\ threshold\ =\ 0.75)}}

Not every candidate polygon corresponds to a real building --- some will
be vegetation clumps, elevated infrastructure, or detection artifacts.
When a reference layer of known building footprints is available (e.g.,
from municipal GIS records or manual digitizing), this function uses
spatial overlap to filter out false detections.

For each candidate polygon, the function computes a \textbf{containment
ratio}: the area of intersection with any overlapping reference polygon
divided by the total area of the candidate. If this ratio meets or
exceeds \texttt{threshold}, the candidate is kept. A ratio of
\texttt{0.75} means that at least 75\% of the detected polygon's area
must fall inside a reference footprint.

Note that this is a one-sided measure --- it evaluates how well the
\emph{candidate} is contained within the reference, not the reverse.
This means a smaller detected polygon that sits neatly inside a large
reference footprint will pass, while a candidate that extends
significantly beyond the reference boundary will be rejected.

\begin{quote}
\textbf{No reference layer available?} Skip this step and proceed
directly to Step 5. The function is optional in the pipeline.
\end{quote}

\textbf{\texttt{threshold}} (default \texttt{0.75}): Lower values
recover more partial matches at the cost of accepting more false
detections; higher values demand tighter spatial agreement with the
reference.

\begin{Shaded}
\begin{Highlighting}[]
\CommentTok{\# Assign a CRS to align all layers (UTM Zone 6N for Fairbanks, AK)}
\NormalTok{terra}\SpecialCharTok{::}\FunctionTok{crs}\NormalTok{(r)        }\OtherTok{\textless{}{-}} \StringTok{"EPSG:32606"}
\NormalTok{r\_crs                }\OtherTok{\textless{}{-}}\NormalTok{ terra}\SpecialCharTok{::}\FunctionTok{crs}\NormalTok{(r)}
\NormalTok{terra}\SpecialCharTok{::}\FunctionTok{crs}\NormalTok{(r\_filled) }\OtherTok{\textless{}{-}}\NormalTok{ r\_crs}
\NormalTok{buildings\_sf         }\OtherTok{\textless{}{-}}\NormalTok{ sf}\SpecialCharTok{::}\FunctionTok{st\_set\_crs}\NormalTok{(buildings\_sf, r\_crs)}
\NormalTok{b                    }\OtherTok{\textless{}{-}}\NormalTok{ sf}\SpecialCharTok{::}\FunctionTok{st\_set\_crs}\NormalTok{(b, r\_crs)}

\NormalTok{filtered }\OtherTok{\textless{}{-}} \FunctionTok{filter\_by\_ground\_truth}\NormalTok{(}
\NormalTok{  buildings\_sf,}
\NormalTok{  b,}
  \AttributeTok{threshold =} \FloatTok{0.5}
\NormalTok{)}

\FunctionTok{ggplot}\NormalTok{() }\SpecialCharTok{+}
  \FunctionTok{geom\_raster}\NormalTok{(}\AttributeTok{data =}\NormalTok{ r\_df, }\FunctionTok{aes}\NormalTok{(}\AttributeTok{x =}\NormalTok{ x, }\AttributeTok{y =}\NormalTok{ y, }\AttributeTok{fill =}\NormalTok{ value)) }\SpecialCharTok{+}
  \FunctionTok{geom\_sf}\NormalTok{(}\AttributeTok{data =}\NormalTok{ filtered, }\AttributeTok{fill =} \ConstantTok{NA}\NormalTok{, }\AttributeTok{color =} \StringTok{"\#c47c0f"}\NormalTok{, }\AttributeTok{linewidth =} \FloatTok{0.6}\NormalTok{) }\SpecialCharTok{+}
  \FunctionTok{coord\_sf}\NormalTok{() }\SpecialCharTok{+}
  \FunctionTok{scale\_fill\_rp}\NormalTok{(}\AttributeTok{name =} \StringTok{"Elev. (m)"}\NormalTok{) }\SpecialCharTok{+}
  \FunctionTok{theme\_rp}\NormalTok{() }\SpecialCharTok{+}
  \FunctionTok{labs}\NormalTok{(}
    \AttributeTok{title    =} \StringTok{"After Ground Truth Filtering"}\NormalTok{,}
    \AttributeTok{subtitle =} \StringTok{"Candidates retained where ≥50\% of their area overlaps a reference footprint"}
\NormalTok{  )}
\end{Highlighting}
\end{Shaded}

\begin{center}\includegraphics{raster-polygonizer-workflow_files/figure-latex/filter-1} \end{center}

\begin{center}\rule{0.5\linewidth}{0.5pt}\end{center}

\subsection{Step 5 --- Remove Invalid
Polygons}\label{step-5-remove-invalid-polygons}

\textbf{\texttt{remove\_invalid\_polys(polys,\ raster,\ buffer\_dist,\ ground\_tol,\ remove\_nested)}}

This function performs two distinct cleanup passes in a deliberate
order.

\textbf{Pass 1 --- Ground contamination check:} For each polygon, the
function extracts the minimum elevation value inside the polygon and the
minimum elevation value in a surrounding buffer ring of width
\texttt{buffer\_dist}. If the absolute difference between these two
minima is less than \texttt{ground\_tol}, the polygon is flagged as
ground-contaminated and removed. The logic is straightforward: a polygon
sitting on an actual rooftop should have notably higher minimum
elevations than the ground immediately surrounding it. If the inside and
outside minima are nearly the same, the polygon is likely sitting on
flat ground rather than a raised surface.

\textbf{Pass 2 --- Nested polygon removal:} After ground filtering, any
polygon that is fully spatially contained within another polygon is
removed. This handles cases where the edge detection and polygonization
steps produce concentric outlines around the same building.
Ground-contaminated polygons are removed \emph{before} this step
intentionally --- otherwise, a large invalid polygon could cause valid
smaller polygons nested within it to be incorrectly discarded.

\textbf{\texttt{ground\_tol}} (default \texttt{3}, map units): The
minimum elevation difference required between inside and outside minima
to consider a polygon a valid raised structure. Increase this in flat
terrain where small elevation differences are common noise; decrease it
where rooftop-to-ground relief is reliably large.

\textbf{\texttt{buffer\_dist}} (default \texttt{5}, map units): Width of
the exterior ring used for the ground comparison. Should be wide enough
to capture true ground-level values beyond the building footprint, but
not so wide that it picks up adjacent buildings.

\begin{Shaded}
\begin{Highlighting}[]
\NormalTok{valid\_polys }\OtherTok{\textless{}{-}} \FunctionTok{remove\_invalid\_polys}\NormalTok{(filtered, r\_filled)}

\FunctionTok{ggplot}\NormalTok{() }\SpecialCharTok{+}
  \FunctionTok{geom\_raster}\NormalTok{(}\AttributeTok{data =}\NormalTok{ r\_df, }\FunctionTok{aes}\NormalTok{(}\AttributeTok{x =}\NormalTok{ x, }\AttributeTok{y =}\NormalTok{ y, }\AttributeTok{fill =}\NormalTok{ value)) }\SpecialCharTok{+}
  \FunctionTok{geom\_sf}\NormalTok{(}\AttributeTok{data =}\NormalTok{ valid\_polys, }\AttributeTok{fill =} \ConstantTok{NA}\NormalTok{, }\AttributeTok{color =} \StringTok{"\#c47c0f"}\NormalTok{, }\AttributeTok{linewidth =} \FloatTok{0.6}\NormalTok{) }\SpecialCharTok{+}
  \FunctionTok{coord\_sf}\NormalTok{() }\SpecialCharTok{+}
  \FunctionTok{scale\_fill\_rp}\NormalTok{(}\AttributeTok{name =} \StringTok{"Elev. (m)"}\NormalTok{) }\SpecialCharTok{+}
  \FunctionTok{theme\_rp}\NormalTok{() }\SpecialCharTok{+}
  \FunctionTok{labs}\NormalTok{(}
    \AttributeTok{title    =} \StringTok{"Validated Building Polygons"}\NormalTok{,}
    \AttributeTok{subtitle =} \StringTok{"Ground{-}contaminated and nested polygons removed"}
\NormalTok{  )}
\end{Highlighting}
\end{Shaded}

\begin{center}\includegraphics{raster-polygonizer-workflow_files/figure-latex/remove-invalid-1} \end{center}

\begin{center}\rule{0.5\linewidth}{0.5pt}\end{center}

\subsection{Step 6a --- Estimate Building
Height}\label{step-6a-estimate-building-height}

\textbf{\texttt{estimate\_building\_height(polys,\ raster,\ buffer\_dist,\ prob,\ clamp)}}

Rather than simply averaging elevation values inside a polygon (which
would give absolute rooftop elevation, not building height), this
function estimates height \emph{relative to the surrounding ground}. For
each polygon it computes a low percentile of raster values inside the
polygon (\texttt{inside\_q}) and the same percentile in a surrounding
buffer ring (\texttt{outside\_q}). Height is then
\texttt{inside\_q\ -\ outside\_q}.

Using a low percentile (default \texttt{prob\ =\ 0.02}, the 2nd
percentile) rather than the minimum makes the estimate robust to rooftop
features like dips or holes. The same percentile is used for both inside
and outside values for consistency. If \texttt{clamp\ =\ TRUE}, negative
heights (which can arise from noisy data or polygon edge effects) are
set to zero. Using as low a value as possible inside our interior
polygon will help to combat bias towards the peak of gabled roofs.

The function returns the input polygon layer with three new columns:
\texttt{inside\_q}, \texttt{outside\_q}, and \texttt{height}.

\textbf{\texttt{buffer\_dist}} (default \texttt{5}, map units): Width of
the exterior ring used to sample surrounding ground elevation. Should
extend far enough to clear the building footprint and capture true
ground values.

\textbf{\texttt{prob}} (default \texttt{0.02}): Quantile used for both
inside and outside elevation estimates. Lower values are more robust to
high-elevation outliers but may be sensitive to data voids if coverage
is thin.

\begin{Shaded}
\begin{Highlighting}[]
\NormalTok{df }\OtherTok{\textless{}{-}} \FunctionTok{estimate\_building\_height}\NormalTok{(valid\_polys, r\_filled)}

\NormalTok{knitr}\SpecialCharTok{::}\FunctionTok{kable}\NormalTok{(}
  \FunctionTok{head}\NormalTok{(df[, }\FunctionTok{c}\NormalTok{(}\StringTok{"building\_id"}\NormalTok{, }\StringTok{"inside\_q"}\NormalTok{, }\StringTok{"outside\_q"}\NormalTok{, }\StringTok{"height"}\NormalTok{)], }\DecValTok{10}\NormalTok{),}
  \AttributeTok{digits  =} \DecValTok{2}\NormalTok{,}
  \AttributeTok{caption =} \StringTok{"Estimated building height for the first 10 polygons. Height is the difference between the 2nd{-}percentile rooftop elevation and the 2nd{-}percentile ground elevation in the surrounding buffer ring."}
\NormalTok{)}
\end{Highlighting}
\end{Shaded}

\begin{longtable}[]{@{}
  >{\raggedright\arraybackslash}p{(\linewidth - 10\tabcolsep) * \real{0.0417}}
  >{\raggedleft\arraybackslash}p{(\linewidth - 10\tabcolsep) * \real{0.1667}}
  >{\raggedleft\arraybackslash}p{(\linewidth - 10\tabcolsep) * \real{0.1250}}
  >{\raggedleft\arraybackslash}p{(\linewidth - 10\tabcolsep) * \real{0.1389}}
  >{\raggedleft\arraybackslash}p{(\linewidth - 10\tabcolsep) * \real{0.0972}}
  >{\raggedright\arraybackslash}p{(\linewidth - 10\tabcolsep) * \real{0.4306}}@{}}
\caption{Estimated building height for the first 10 polygons. Height is
the difference between the 2nd-percentile rooftop elevation and the
2nd-percentile ground elevation in the surrounding buffer
ring.}\tabularnewline
\toprule\noalign{}
\begin{minipage}[b]{\linewidth}\raggedright
\end{minipage} & \begin{minipage}[b]{\linewidth}\raggedleft
building\_id
\end{minipage} & \begin{minipage}[b]{\linewidth}\raggedleft
inside\_q
\end{minipage} & \begin{minipage}[b]{\linewidth}\raggedleft
outside\_q
\end{minipage} & \begin{minipage}[b]{\linewidth}\raggedleft
height
\end{minipage} & \begin{minipage}[b]{\linewidth}\raggedright
geometry
\end{minipage} \\
\midrule\noalign{}
\endfirsthead
\toprule\noalign{}
\begin{minipage}[b]{\linewidth}\raggedright
\end{minipage} & \begin{minipage}[b]{\linewidth}\raggedleft
building\_id
\end{minipage} & \begin{minipage}[b]{\linewidth}\raggedleft
inside\_q
\end{minipage} & \begin{minipage}[b]{\linewidth}\raggedleft
outside\_q
\end{minipage} & \begin{minipage}[b]{\linewidth}\raggedleft
height
\end{minipage} & \begin{minipage}[b]{\linewidth}\raggedright
geometry
\end{minipage} \\
\midrule\noalign{}
\endhead
\bottomrule\noalign{}
\endlastfoot
3 & 3 & 152.32 & 145.34 & 6.98 & POLYGON ((465880.9 7193013,\ldots{} \\
4 & 4 & 149.50 & 144.93 & 4.57 & POLYGON ((465784.9 7193026,\ldots{} \\
5 & 5 & 151.92 & 145.17 & 6.75 & POLYGON ((465886.5 7193049,\ldots{} \\
6 & 6 & 155.17 & 145.01 & 10.16 & POLYGON ((465708.5 7193103,\ldots{} \\
7 & 7 & 149.44 & 145.21 & 4.23 & POLYGON ((465757.5 7193110,\ldots{} \\
9 & 9 & 153.41 & 145.68 & 7.73 & POLYGON ((465884.5 7193135,\ldots{} \\
10 & 10 & 151.50 & 144.83 & 6.67 & POLYGON ((465728.5
7193239,\ldots{} \\
12 & 12 & 154.38 & 144.96 & 9.42 & POLYGON ((465919.5
7193279,\ldots{} \\
13 & 13 & 154.36 & 144.85 & 9.51 & POLYGON ((465809.5
7193281,\ldots{} \\
14 & 14 & 154.42 & 144.88 & 9.55 & POLYGON ((465770.5
7193333,\ldots{} \\
\end{longtable}

\begin{center}\rule{0.5\linewidth}{0.5pt}\end{center}

\subsection{Step 6b --- Roof Slope and Aspect via
RANSAC}\label{step-6b-roof-slope-and-aspect-via-ransac}

\textbf{\texttt{roof\_slope\_RANSAC(raster,\ buildings,\ n\_iter,\ thresh,\ min\_inliers)}}

Roof slope and orientation (aspect) are key determinants of snow
retention --- a steep south-facing roof sheds snow quickly, while a
shallow north-facing roof may accumulate loads several times greater.
This function fits a plane to the raster elevation values within each
polygon using RANSAC (Random Sample Consensus), a robust estimation
method that explicitly tolerates outlier pixels caused by dormers,
vents, parapets, and sensor noise.

Each RANSAC iteration randomly draws 3 pixels from the building's raster
footprint, fits a plane through them via linear regression
(\texttt{z\ \textasciitilde{}\ x\ +\ y}), and counts how many of the
\emph{remaining} pixels have residuals below \texttt{thresh} (inliers).
After \texttt{n\_iter} iterations, the plane that maximized inlier count
is selected and used to compute final slope and aspect:

\begin{itemize}
\tightlist
\item
  \textbf{Slope} = \texttt{atan(sqrt(a²\ +\ b²))\ ×\ 180/π}, where
  \texttt{a} and \texttt{b} are the x and y plane coefficients
\item
  \textbf{Aspect} = \texttt{(atan2(-a,\ -b)\ ×\ 180/π)\ \%\%\ 360},
  giving compass bearing of the downslope direction
\end{itemize}

A \textbf{stability metric} (\texttt{slope\_range}) is also returned:
the 97.5th minus 2.5th percentile of slope estimates across the top 5\%
of candidate fits, ranked by inlier count. A low value means those top
candidates agreed closely --- the roof surface is planar and the fit is
reliable. A high value suggests the roof geometry is complex or the data
is noisy.

\textbf{\texttt{thresh}} (default \texttt{0.1}, elevation units): The
residual distance within which a pixel is counted as an inlier. Tighten
this for high-density LiDAR where rooftop surfaces are reliably smooth.
Loosen it for coarser data where pixel-to-plane scatter is inherently
larger. \emph{For scenes with many small buildings, individual polygons
contain fewer pixels, so each pixel carries more weight --- a tighter
\texttt{thresh} here helps ensure the fitted plane is genuinely
representative of the roof surface rather than a statistical accident.}

\textbf{\texttt{min\_inliers}} (default \texttt{10}): Minimum pixels
required to accept any candidate plane. For small buildings with few
raster cells, or lower resolution rasters, lower this to avoid
discarding valid fits. For large buildings, raising it ensures the plane
is well-supported across the full footprint. Mostly this parameter is
treated as a fail-safe, and is not meant to be tuned.

\textbf{\texttt{n\_iter}} (default \texttt{500}): Number of random
3-point samples evaluated. More iterations improve the chance of finding
the globally best plane. Raise this towards the ceiling of what
computational power will allow for best-fit estimates.

\begin{Shaded}
\begin{Highlighting}[]
\NormalTok{buildings\_sv }\OtherTok{\textless{}{-}}\NormalTok{ terra}\SpecialCharTok{::}\FunctionTok{vect}\NormalTok{(valid\_polys)}

\ControlFlowTok{if}\NormalTok{ (}\SpecialCharTok{!}\StringTok{"building\_id"} \SpecialCharTok{\%in\%} \FunctionTok{names}\NormalTok{(buildings\_sv)) \{}
\NormalTok{  buildings\_sv}\SpecialCharTok{$}\NormalTok{building\_id }\OtherTok{\textless{}{-}} \FunctionTok{seq\_len}\NormalTok{(}\FunctionTok{nrow}\NormalTok{(buildings\_sv))}
\NormalTok{\}}

\NormalTok{ransac\_result }\OtherTok{\textless{}{-}} \FunctionTok{roof\_slope\_RANSAC}\NormalTok{(}
  \AttributeTok{raster      =}\NormalTok{ r\_filled,}
  \AttributeTok{buildings   =}\NormalTok{ buildings\_sv,}
  \AttributeTok{n\_iter      =} \DecValTok{500}\DataTypeTok{L}\NormalTok{,}
  \AttributeTok{thresh      =} \FloatTok{0.1}\NormalTok{,}
  \AttributeTok{min\_inliers =} \DecValTok{10}\DataTypeTok{L}\NormalTok{,}
  \AttributeTok{quiet       =} \ConstantTok{TRUE}
\NormalTok{)}

\NormalTok{knitr}\SpecialCharTok{::}\FunctionTok{kable}\NormalTok{(}
  \FunctionTok{head}\NormalTok{(ransac\_result}\SpecialCharTok{$}\NormalTok{summary\_table, }\DecValTok{10}\NormalTok{),}
  \AttributeTok{digits  =} \DecValTok{2}\NormalTok{,}
  \AttributeTok{caption =} \StringTok{"Roof slope and aspect estimates. slope\_range is the spread of slope values across the top 5\% of RANSAC candidate fits — lower values indicate a more stable, reliable plane fit."}
\NormalTok{)}
\end{Highlighting}
\end{Shaded}

\begin{longtable}[]{@{}
  >{\raggedleft\arraybackslash}p{(\linewidth - 10\tabcolsep) * \real{0.1600}}
  >{\raggedleft\arraybackslash}p{(\linewidth - 10\tabcolsep) * \real{0.1333}}
  >{\raggedleft\arraybackslash}p{(\linewidth - 10\tabcolsep) * \real{0.1467}}
  >{\raggedleft\arraybackslash}p{(\linewidth - 10\tabcolsep) * \real{0.1333}}
  >{\raggedleft\arraybackslash}p{(\linewidth - 10\tabcolsep) * \real{0.1600}}
  >{\raggedleft\arraybackslash}p{(\linewidth - 10\tabcolsep) * \real{0.2667}}@{}}
\caption{Roof slope and aspect estimates. slope\_range is the spread of
slope values across the top 5\% of RANSAC candidate fits --- lower
values indicate a more stable, reliable plane fit.}\tabularnewline
\toprule\noalign{}
\begin{minipage}[b]{\linewidth}\raggedleft
building\_id
\end{minipage} & \begin{minipage}[b]{\linewidth}\raggedleft
slope\_deg
\end{minipage} & \begin{minipage}[b]{\linewidth}\raggedleft
aspect\_deg
\end{minipage} & \begin{minipage}[b]{\linewidth}\raggedleft
n\_inliers
\end{minipage} & \begin{minipage}[b]{\linewidth}\raggedleft
slope\_range
\end{minipage} & \begin{minipage}[b]{\linewidth}\raggedleft
top\_5\_percent\_count
\end{minipage} \\
\midrule\noalign{}
\endfirsthead
\toprule\noalign{}
\begin{minipage}[b]{\linewidth}\raggedleft
building\_id
\end{minipage} & \begin{minipage}[b]{\linewidth}\raggedleft
slope\_deg
\end{minipage} & \begin{minipage}[b]{\linewidth}\raggedleft
aspect\_deg
\end{minipage} & \begin{minipage}[b]{\linewidth}\raggedleft
n\_inliers
\end{minipage} & \begin{minipage}[b]{\linewidth}\raggedleft
slope\_range
\end{minipage} & \begin{minipage}[b]{\linewidth}\raggedleft
top\_5\_percent\_count
\end{minipage} \\
\midrule\noalign{}
\endhead
\bottomrule\noalign{}
\endlastfoot
3 & 17.84 & 95.44 & 25 & 2.33 & 11 \\
4 & 1.83 & 181.88 & 38 & 1.18 & 11 \\
5 & 5.01 & 180.47 & 234 & 0.66 & 16 \\
6 & 0.34 & 203.79 & 2130 & 0.35 & 23 \\
7 & 0.04 & 12.72 & 218 & 0.35 & 22 \\
9 & 0.61 & 303.03 & 742 & 0.57 & 24 \\
10 & 0.32 & 354.93 & 693 & 0.46 & 20 \\
12 & 0.05 & 199.72 & 964 & 0.18 & 24 \\
13 & 0.24 & 267.55 & 932 & 0.25 & 24 \\
14 & 1.09 & 90.90 & 911 & 0.37 & 25 \\
\end{longtable}

\begin{center}\rule{0.5\linewidth}{0.5pt}\end{center}

\subsection{What's Next?}\label{whats-next}

With per-building height, slope, and aspect values in hand, the natural
next steps are:

\begin{itemize}
\tightlist
\item
  \textbf{Snow load estimation} --- difference two DSMs (pre- and
  post-snowfall) within each validated polygon to derive snow depth,
  then combine with slope to estimate structural load
\item
  \textbf{Spatial aggregation} --- summarize risk metrics by
  neighborhood, building type, or construction era
\item
  \textbf{Validation} --- compare derived slope values against as-built
  drawings or field measurements to quantify method accuracy
\end{itemize}

All outputs are standard R objects --- \texttt{sf} data frames and
\texttt{terra} rasters --- and can be written to disk with
\texttt{sf::st\_write()} and \texttt{terra::writeRaster()} respectively.

\end{document}
